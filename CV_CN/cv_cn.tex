\documentclass[11pt,a4paper,roman]{moderncv}        % possible options include font size ('10pt', '11pt' and '12pt'), paper size ('a4paper', 'letterpaper', 'a5paper', 'legalpaper', 'executivepaper' and 'landscape') and font family ('sans' and 'roman')

% modern themes
\moderncvstyle{banking}                            % style options are 'casual' (default), 'classic', 'oldstyle' and 'banking'
\moderncvcolor{blue}                                % color options 'blue' (default), 'orange', 'green', 'red', 'purple', 'grey' and 'black'
%\renewcommand{\familydefault}{\sfdefault}         % to set the default font; use '\sfdefault' for the default sans serif font, '\rmdefault' for the default roman one, or any tex font name
\nopagenumbers{}                                  % uncomment to suppress automatic page numbering for CVs longer than one page

% character encoding
\usepackage[utf8]{inputenc}
\usepackage{fontawesome}
\usepackage{tabularx}
\usepackage{ragged2e}
% if you are not using xelatex ou lualatex, replace by the encoding you are using
%\usepackage{CJKutf8}                              % if you need to use CJK to typeset your resume in Chinese, Japanese or Korean

% adjust the page margins
\usepackage[scale=0.89]{geometry}
\usepackage{multicol}
%\setlength{\hintscolumnwidth}{3cm}                % if you want to change the width of the column with the dates

\usepackage{import}
\usepackage{multicol}
\usepackage{ctex} 

% personal data

\name{梁}{浩}
\title{个人简历}                          


\newcommand*{\customcventry}[7][.25em]{
	\begin{tabular}{@{}l} 
		{\bfseries #4}
	\end{tabular}
	\hfill% move it to the right
	\begin{tabular}{l@{}}
		{\bfseries #5}
	\end{tabular} \\
	\begin{tabular}{@{}l} 
		{\itshape #3}
	\end{tabular}
	\hfill% move it to the right
	\begin{tabular}{l@{}}
		{\itshape #2}
	\end{tabular}
	\ifx&#7&%
	\else{\\%
		\begin{minipage}{\maincolumnwidth}%
			\small#7%
	\end{minipage}}\fi%
	\par\addvspace{#1}}

\newcommand*{\customcvproject}[4][.25em]{
	%   \vfill\noindent
	\begin{tabular}{@{}l} 
		{\bfseries #2}
	\end{tabular}
	\hfill% move it to the right
	\begin{tabular}{l@{}}
		{\itshape #3}
	\end{tabular}
	\ifx&#4&%
	\else{\\%
		\begin{minipage}{\maincolumnwidth}%
			\small#4%
	\end{minipage}}\fi%
	\par\addvspace{#1}}

\setlength{\tabcolsep}{12pt}

%----------------------------------------------------------------------------------
%            content
%----------------------------------------------------------------------------------
\begin{document}
	%\begin{CJK*}{UTF8}{gbsn}                          % to typeset your resume in Chinese using CJK
	%-----       resume       ---------------------------------------------------------
	\makecvtitle
	\vspace*{-16mm}
	
	\begin{center}
		\renewcommand{\arraystretch}{1.3}
		\begin{tabular}{ c c c c }
			\faGithub\enspace github.com/hauliang \  \enspace \faGlobe\enspace hauliang.github.io  \\
			\faEnvelopeO\enspace haoliang@stu.xmu.edu.cn \enspace  \enspace \faMobile\enspace (+86) 13602597958 \\
		\end{tabular}
	\end{center}
	
	\vspace*{-2.5mm}
	
	%\renewcommand{\arraystretch}{1.5}

	\cfoot{\vspace{-2mm}{\color{gray} \rule[-10pt]{14.3cm}{0.05em}} \vspace{2mm}\\ \textcolor{gray}{最后更新时间:2022/08/05}}
	
	\section{研究兴趣}
	非平稳信号分解,声学波束成形,信号处理 \\
	稀疏优化理论,机器学习,模型驱动的深度学习
	
	\section{教育经历}
	{\customcventry{2020.09 - 2023.07 (预计)}{信号与信息处理专业工学硕士 }{厦门大学}{福建厦门,中国}{}{}}
	
	\begin{itemize}
		\item[--] 绩点:3.80/4.0  
		\item[--] 排名:1/104 
	\end{itemize}
	\vspace{2mm}
	
	{\customcventry{2016.09 - 2020.07}{电子信息工程专业工学学士,统计学专业理学学士(双学位)}{深圳大学}{广东深圳,中国}{}{}}
	
	\begin{itemize}
		\item[--] 绩点:3.86/4.5 
		\item[--] 排名:6/155 
		\item[--] 均分:89.2 
	\end{itemize}

	
	\section{奖项 \& 荣誉}
	\begin{minipage}{\maincolumnwidth}%
			\begin{itemize}
				\item \textbf{优秀学生干部},厦门大学,2022
				\vspace{1mm}
				\item \textbf{校级奖学金(一等)},厦门大学, 2022
				\vspace{1mm}
				\item \textbf{优秀志愿者},厦门大学,2022
				\vspace{1mm}
				\item \textbf{优秀三好学生},厦门大学,2021
				\vspace{1mm}
				\item \textbf{优秀毕业生},深圳大学,2020
				\vspace{1mm}
				\item \textbf{校级奖学金(一等)},深圳大学,2019
				\vspace{1mm}
				\item \textbf{学业奖学金(二等)},深圳大学,2019
				\vspace{1mm}
				\item \textbf{学业奖学金(二等)},深圳大学,2017
		\end{itemize}%
	\end{minipage}%
	\vspace*{1.5mm}
	
	\section{科研经历}
	{\customcvproject{计算声成像}{2020 - 2022} 
		{\begin{itemize}
				\vspace*{1mm}
				\item 本项目侧重于阵列信号处理、波束成形和模型驱动的深度学习。 \vspace*{0.75mm}
				\item 本项目通过波束形成算法,对麦克风阵列采集的多维阵列信号进行处理,进而生成扫描平面上的声压级分布,再通过照片或视频的形式将“声音可视化”。 \vspace*{0.75mm}
				\item 本项目对于噪声监测、工业故障诊断等领域有着重要作用。
			\end{itemize}
		}
	}
	
	\vspace*{1.2mm}
	
	{\customcvproject{自适应稀疏时频估计}{2020 – 2022}
		{\begin{itemize}
				\vspace*{1mm}
				\item 本项目侧重于稀疏时频分解和稀疏贝叶斯学习。 \vspace*{0.75mm}
				\item 本项目的目标为自适应地估计非平稳信号(例如非线性啁啾信号和色散信号)的各个参数(例如瞬时幅度和频率),然后以数据驱动的方式学习重构字典,从而实现高分辨的时频表示结果。 \vspace*{0.75mm}
				\item 本项目有助于提升信号时频分析的理论价值和故障诊断的应用前景。
			\end{itemize}
		}
	}


	\section{专业技能}
	\begin{tabular}{ @{} >{\bfseries}l @{\hspace{6ex}} l }
		编程\ & Matlab, LaTex, Python, Pytorch \\
		语言\ & 中文,英文  \\
		软件\ & Office, Endnote \\
	\end{tabular}

	
	
	\section{论文发表}
	\begin{itemize}
		\item \textbf{High-Resolution Source Localization Exploiting the Sparsity of the Beamforming Map} \\
		Ding, Xinghao and \textbf{Liang, Hao} and Jakobsson, Andreas and Tu, Xiaotong* and Huang, Yue \\
		\emph{Signal Processing}, 2022. \textbf{IF: 4.729, Rank: Q1.} 
		\vspace{2mm}
		\item \textbf{A Robust Low-Rank Matrix Completion Based on Truncated Nuclear Norm and Lp-norm} \\
		\textbf{Liang, Hao} and Li, Kang* and Huang, Jianjun \\
		\emph{The Journal of Supercomputing}, 2022. \textbf{IF: 2.557, Rank: Q2.}
		\vspace{2mm}
		\item \textbf{Adaptive Variational Nonlinear Chirp Mode Decomposition} \\
		\textbf{Liang, Hao} and Ding, Xinghao and Jakobsson, Andreas and Tu, Xiaotong* and Huang, Yue \\
		in \emph{2022 IEEE International Conference on Acoustics, Speech and Signal Processing (ICASSP)}, 2022.
	\end{itemize}

	\section{其他论文(审稿中)}
	
	\begin{itemize}
		\item \textbf{Sparse Optimization for Nonlinear Group Delay Mode Estimation} \\
		\textbf{Liang, Hao} and Ding, Xinghao and Jakobsson, Andreas and Tu, Xiaotong* and Huang, Yue \\
		submitted to \emph{The Journal of the Acoustical Society of America}, 2022. \textbf{IF: 2.482, Rank: Q2.} 
	\end{itemize}
	
	
	
	\section{专业课程}
	
	\begin{itemize}
		\item \textbf{研究生阶段:} \vspace*{0.75mm} \\
		机器学习 (93),矩阵论 (88),最优化理论与工程应用 (93) \\
		随机过程 (94),数字信号处理 (88),应用信息论 (90)
		\vspace*{1.75mm}
		\item \textbf{本科阶段:} \vspace*{0.75mm} \\
		线性代数 (99),高等数学A1 (95),高等数学A2 (90),场论与复变函数 (94),信号与系统 (86)\\
		数值理论与计算方法 (90),随机信号分析 (97),数字信号处理 (96),概率论与数理统计 (94)\\
		数字图像处理 (88),高等代数1 (95),数学分析2 (96),随机过程 (99),运筹学 (94),高等代数2 (93)\\
		数学分析3 (99),非参数统计 (90),贝叶斯统计 (96)
 	\end{itemize}
	
	\section{推荐老师}
	
	\vspace*{-3.5mm}
	
	\begin{multicols}{2}
		\begin{itemize}
			\item \textbf{丁兴号}\\
			厦门大学信息学院教授 \\
			邮箱: dxh@xmu.edu.cn
		\end{itemize}
		
		\begin{itemize}
			\item \textbf{涂晓彤}\\
			厦门大学信息学院助理教授\\
			邮箱: xttu@xmu.edu.cn
		\end{itemize}
	\end{multicols}
	

%	\begin{itemize}
%		\item \textbf{Dr. Xinghao Ding}\\
%		Professor, School of Informatics\\
%		Xiamen University, Xiamen, CHN \\
%		Email: dxh@xmu.edu.cn
%		
%		\vspace{2mm}
%		\item \textbf{Dr. Xiaotong Tu}\\
%		Assistant Professor, School of Informatics\\
%		Xiamen University, Xiamen, CHN\\
%		Email: xttu@xmu.edu.cn
%		
%	\end{itemize}
	
	
	

	

	\cfoot{\vspace{-2mm}{\color{gray} \rule[-10pt]{14.3cm}{0.05em}} \vspace{2mm}\\ \textcolor{gray}{最后更新时间:2022/08/05}}
	
	
\end{document}


%% end of file `template.tex'.
